% This is "sig-alternate.tex" V2.1 April 2013
% This file should be compiled with V2.5 of "sig-alternate.cls" May 2012
%
% This example file demonstrates the use of the 'sig-alternate.cls'
% V2.5 LaTeX2e document class file. It is for those submitting
% articles to ACM Conference Proceedings WHO DO NOT WISH TO
% STRICTLY ADHERE TO THE SIGS (PUBS-BOARD-ENDORSED) STYLE.
% The 'sig-alternate.cls' file will produce a similar-looking,
% albeit, 'tighter' paper resulting in, invariably, fewer pages.
%
% ----------------------------------------------------------------------------------------------------------------
% This .tex file (and associated .cls V2.5) produces:
%       1) The Permission Statement
%       2) The Conference (location) Info information
%       3) The Copyright Line with ACM data
%       4) NO page numbers
%
% as against the acm_proc_article-sp.cls file which
% DOES NOT produce 1) thru' 3) above.
%
% Using 'sig-alternate.cls' you have control, however, from within
% the source .tex file, over both the CopyrightYear
% (defaulted to 200X) and the ACM Copyright Data
% (defaulted to X-XXXXX-XX-X/XX/XX).
% e.g.
% \CopyrightYear{2007} will cause 2007 to appear in the copyright line.
% \crdata{0-12345-67-8/90/12} will cause 0-12345-67-8/90/12 to appear in the copyright line.
%
% ---------------------------------------------------------------------------------------------------------------
% This .tex source is an example which *does* use
% the .bib file (from which the .bbl file % is produced).
% REMEMBER HOWEVER: After having produced the .bbl file,
% and prior to final submission, you *NEED* to 'insert'
% your .bbl file into your source .tex file so as to provide
% ONE 'self-contained' source file.
%
% ================= IF YOU HAVE QUESTIONS =======================
% Questions regarding the SIGS styles, SIGS policies and
% procedures, Conferences etc. should be sent to
% Adrienne Griscti (griscti@acm.org)
%
% Technical questions _only_ to
% Gerald Murray (murray@hq.acm.org)
% ===============================================================
%
% For tracking purposes - this is V2.0 - May 2012

\documentclass{sig-alternate-05-2015}
\usepackage[utf8]{inputenc}
\usepackage[portuguese]{babel} 
\usepackage{algpseudocode,algorithm}
% Declaracoes em Português
\algrenewcommand\algorithmicend{\textbf{fim}}
\algrenewcommand\algorithmicdo{\textbf{faça}}
\algrenewcommand\algorithmicwhile{\textbf{enquanto}}
\algrenewcommand\algorithmicfor{\textbf{para}}
\algrenewcommand\algorithmicif{\textbf{se}}
\algrenewcommand\algorithmicthen{\textbf{então}}
\algrenewcommand\algorithmicelse{\textbf{senão}}
\algrenewcommand\algorithmicreturn{\textbf{devolve}}
\algrenewcommand\algorithmicfunction{\textbf{função}}

% Rearranja os finais de cada estrutura
\algrenewtext{EndWhile}{\algorithmicend\ \algorithmicwhile}
\algrenewtext{EndFor}{\algorithmicend\ \algorithmicfor}
\algrenewtext{EndIf}{\algorithmicend\ \algorithmicif}
\algrenewtext{EndFunction}{\algorithmicend\ \algorithmicfunction}

% O comando For, a seguir, retorna 'para #1 -- #2 até #3 faça'
\algnewcommand\algorithmicto{\textbf{até}}
\algrenewtext{For}[3]%
{\algorithmicfor\ #1 $\gets$ #2 \algorithmicto\ #3 \algorithmicdo}
\begin{document}

% Copyright
\setcopyright{acmcopyright}
%\setcopyright{acmlicensed}
%\setcopyright{rightsretained}
%\setcopyright{usgov}
%\setcopyright{usgovmixed}
%\setcopyright{cagov}
%\setcopyright{cagovmixed}


%
%\doi{10.475/123_4}

%% ISBN

% --- End of Author Metadata ---

\title{Estudo do Problema de Árvores Geradoras de Rótulos Mínimos\titlenote{(Produces the permission block, and
copyright information). For use 
SIG-ALTERNATE.CLS. Supported by ACM.}}
\subtitle{[Implementação e análise de instâncias usando JPSO]
\titlenote{A full version of this paper is available as
\textit{Author's Guide to Preparing ACM SIG Proceedings Using
\LaTeX$2_\epsilon$\ and BibTeX} at
\texttt{www.acm.org/eaddress.htm}}}
%
% You need the command \numberofauthors to handle the 'placement
% and alignment' of the authors beneath the title.
%
% For aesthetic reasons, we recommend 'three authors at a time'
% i.e. three 'name/affiliation blocks' be placed beneath the title.
%
% NOTE: You are NOT restricted in how many 'rows' of
% "name/affiliations" may appear. We just ask that you restrict
% the number of 'columns' to three.
%
% Because of the available 'opening page real-estate'
% we ask you to refrain from putting more than six authors
% (two rows with three columns) beneath the article title.
% More than six makes the first-page appear very cluttered indeed.
%
% Use the \alignauthor commands to handle the names
% and affiliations for an 'aesthetic maximum' of six authors.
% Add names, affiliations, addresses for
% the seventh etc. author(s) as the argument for the
% \additionalauthors command.
% These 'additional authors' will be output/set for you
% without further effort on your part as the last section in
% the body of your article BEFORE References or any Appendices.

\numberofauthors{2} %  in this sample file, there are a *total*
% of EIGHT authors. SIX appear on the 'first-page' (for formatting
% reasons) and the remaining two appear in the \additionalauthors section.
%
\author{
% You can go ahead and credit any number of authors here,
% e.g. one 'row of three' or two rows (consisting of one row of three
% and a second row of one, two or three).
%
% The command \alignauthor (no curly braces needed) should
% precede each author name, affiliation/snail-mail address and
% e-mail address. Additionally, tag each line of
% affiliation/address with \affaddr, and tag the
% e-mail address with \email.
%
% 1st. author
\alignauthor
 Paulo Batista da Costa\\
       \affaddr{Universidade Tecnológica Federal do Paraná}\\
       \affaddr{2015 Campo Mourão}\\
       \affaddr{Campo Mourão, Brasil}\\
       \email{pauloc@alunos.utfpr.edu.br}
% 2nd. author
\alignauthor
		Felipe Veiga Ramos\\
       \affaddr{Universidade Tecnológica Federal do Paraná}\\
       \affaddr{2015 Campo Mourão}\\
       \affaddr{Peabiru, Brasil}\\
       \email{fveigaramos@gmail.com}
}
% There's nothing stopping you putting the seventh, eighth, etc.
% author on the opening page (as the 'third row') but we ask,
% for aesthetic reasons that you place these 'additional authors'
% in the \additional authors block, viz.
%\additionalauthors{Additional authors: John Smith (The Th{\o}rv{\"a}ld Group,
%email: {\texttt{jsmith@affiliation.org}}) and Julius P.~Kumquat
%(The Kumquat Consortium, email: {\texttt{jpkumquat@consortium.net}}).}
%\date{30 July 1999}
% Just remember to make sure that the TOTAL number of authors
% is the number that will appear on the first page PLUS the
% number that will appear in the \additionalauthors section.

\maketitle
\begin{abstract}
Um dos assuntos fundamentais da área da Teoria dos grafos são as árvores geradoras de rótulos mínimos. Elas tem por objetivo encontrar o menor número possível de rótulos de um grafo não dirigido em conexo que cubram todos os vértices possíveis. Os problemas de árvores geradoras de rótulos mínimos ou PAGRM são de grande importância para áreas que projetam redes de comunicação, elétrica, transporte e dentre outros. Desse modo, para estudar e ampliar os conhecimentos na área este trabalho foi proposto. Ele  consiste na implementação de um algoritmo -- no caso o JPSO -- e na sua comparação com demais algoritmos propostos pela literatura de acordo com uma base comum. Assim, este trabalho é uma oportunidade de compreender e estudar na prática conteúdos que envolvem teoria dos grafos.
\end{abstract}


%
% The code below should be generated by the tool at
% http://dl.acm.org/ccs.cfm
% Please copy and paste the code instead of the example below. 
%
\begin{CCSXML}
<ccs2012>
 <concept>
  <concept_id>10010520.10010553.10010562</concept_id>
  <concept_desc>Computer systems organization~Embedded systems</concept_desc>
  <concept_significance>500</concept_significance>
 </concept>
 <concept>
  <concept_id>10010520.10010575.10010755</concept_id>
  <concept_desc>Computer systems organization~Redundancy</concept_desc>
  <concept_significance>300</concept_significance>
 </concept>
 <concept>
  <concept_id>10010520.10010553.10010554</concept_id>
  <concept_desc>Computer systems organization~Robotics</concept_desc>
  <concept_significance>100</concept_significance>
 </concept>
 <concept>
  <concept_id>10003033.10003083.10003095</concept_id>
  <concept_desc>Networks~Network reliability</concept_desc>
  <concept_significance>100</concept_significance>
 </concept>
</ccs2012>  
\end{CCSXML}

\ccsdesc[500]{Computer systems organization~Embedded systems}
\ccsdesc[300]{Computer systems organization~Redundancy}
\ccsdesc{Computer systems organization~Robotics}
\ccsdesc[100]{Networks~Network reliability}


%
% End generated code
%

%
%  Use this command to print the description
%
\printccsdesc

% We no longer use \terms command
%\terms{Theory}

\keywords{ACM proceedings; \LaTeX; text tagging}

\section{Introdução}

Um dos assuntos fundamentais da área de Teoria dos grafos são as árvores geradoras de rótulos mínimos. Esta pode ser entendida como uma árvore geradora obtida através de um grafo conectado não direcionado \cite{}. Os Problemas das árvores geradoras de rótulos mínimos (PAGRM)  podem ser formalmente definidos como: ``Dado um grafo rotulado não dirigido $G = (V,E,L) $ no qual $ V $ é o conjunto $ n$ de vértices, $E$ é o conjunto de $m$ arestas e $L$ é o conjunto de $l$ rótulos, encontrar uma árvore geradora $T$ de $G$ tal que $|L_{T}|$ é minimizado. Define-se  $|L_{T}|$ como o conjunto de diferentes rótulos das arestas em uma árvore geradora $T$."

Tal é utilizado para representar situações presentes no mundo real como redes de comunicação, elétrica, transporte e dentre outros. Sua compreensão e estudo é fundamental para a elaboração de soluções que visam reduzir custos e ampliar eficiência. Em outras palavras, o objetivo do estudo e aplicações de algoritmos para resolver o PAGRM é gerar - a partir de um grafo não dirigido - uma árvore que contenha o menor número possível de rótulos e ao mesmo tempo cobrindo todos os seus vértices.

Este trabalho tem por finalidade analisar o conjunto de instâncias  -- o mesmo conjunto utilizado em \cite{} -- através da implementação (realizada na linguagem de programação JAVA) e execução do algoritmo JPSO (\textit{Jumping Particle Swarm Optimization}) \cite{}, e dessa forma estabelecer uma comparação entre os resultados obtidos desta aplicação e aqueles provindos das heurísticas apresentadas em \cite{}. 

Tal comparação levou em consideração fatores como a complexidade dos grafos (número de vértices e rótulos) , valor médio da função objetivo e o tempo computacional. Dessa forma, os resultados são comparados da mesma forma como reportado em \cite{}.

O restante deste trabalho está organizado em seções da seguinte forma: Na seção \ref{sec2} é realizada uma revisão literária, na \ref{sec3} é apresentada a proposta de algoritmo exato para solução do PAGRM (no caso o JPSO), em \ref{sec4} os resultados da implementação são mostrados, em \ref{sec5} é explicitada uma comparação entre os resultados da literatura e os obtidos neste trabalho e por último são apresentadas as conclusões em \ref{sec6}.



\section{Revisão Literária} \label{sec2}


\section{Proposta de Algoritmo -- O JPSO} \label{sec3}

\subsection{Heurística - A ideia do JPSO}

O PSO (Do inglês \textit{Particle Swarm Optmization}) é um algoritmo que simula o comportamento de revoada de indivíduos na natureza. Cada solução do problema  é tido como um indivíduo que pode voar dentro do espaço de soluções. A melhoria das posições, ou seja, as melhores respostas são obtidas através da continua movimentação das partículas que constituem tal revoada. É um algoritmo proposto por \cite{} que utiliza o conceito de inteligência coletiva e é capaz de auxiliar na busca por respostas em problemas de PAGRM. Entretanto, esse algoritmo é formulado para resolver problemas contínuos, o que contradiz o uso para tal. Isto pois, um PAGRM constitui um cenário de resolução discreta.


Portanto, para a utilização da ideia provinda do PSO  é necessário que o algoritmo foque problemas discretos. Assim, uma adaptação é utilizada: O JPSO (\textit{Jumping Particle Swarm Optimization}).
O JPSO  é um DPSO (Do inglês,\textit{Discrete Particle Swarm Optimization}) \cite{} , ou seja, possui 
a mesma essência da heurística aplicada pelo PSO com a diferença de ser voltado para problemas discretos. A ideia é que a cada iteração as possíveis resposta estejam mais próximas do que seria considerado como ótimo. 

\subsection{Explicações da implementação do JPSO}
A implementação para este trabalho está disposta na linguagem de programação JAVA  e está de acordo com o pseudo-código descrito em \cite{}. Assim, a entrada para o JPSO é um grafo não dirigido e conexo $G = (V,E,L)$, com $n$ vértices, 	$m$ arestas, $l$ rótulos e $Q \subseteq V$ nós. Além disto, tem como saída uma árvore geradora a qual objetiva ser mínima, ou seja, conter o menor número de rótulos possível e ao mesmo tempo cobrir todos os vértices do grafo. 

Inicialmente foram carregadas as instâncias presentes nos arquivos disponibilizados em ambiente virtual referente aos grafos sobre os quais a análise foi aplicada e que são as  mesmas utilizadas por \cite{}. Estas foram carregadas em estruturas as quais retornam um grafo não dirigido e conexo provindo da base de dados. 
	
Em seguida, a partir de cada grafo abstraído da base de dados foram geradas 100 partículas. Estas partículas são a representação das possíveis soluções e são inicialmente formadas de forma randômica: Para cada partícula é induzido um grafo aleatoriamente a partir dos rótulos do grafo original.

A partir desse momento iniciam-se as iterações que buscam a melhor resposta, elas são caracterizadas pelos seguintes passos: 1. atualiza-se a melhor resposta (componente conexa com menor número de rótulos) para cada partícula , 2. atualiza-se a melhor resposta global já obtida, 3. Para cada partícula há a tentativa de reduzir o número de rótulos mantendo as propriedades necessárias, 4. tentativa de combinar duas soluções gerando um grafo com menor número de rótulos.

As tentativas de modificações de cada partículas são randômicas. Além disso é importante ressaltar que a cada iteração as respostas possíveis nunca apresentarão maior número de rótulos do que na iteração anterior, pois toda vez que ocorrer aumento no número de rótulos há o descarte da solução mantendo a anterior. Outro ponto importante é o critério de parada para o algoritmo:É finalizado a partir do momento em que se completam 100 iterações sem modificação no melhor resultado global.

Assim, o JPSO deve retornar a melhor resposta obtida durante sua execução. Apesar de garantir o retorno da melhor solução, não garante que esta seja a solução ótima. Em outras palavras, não retorna necessariamente a arvore com menor rótulos possível de ser formada. Isto devido ao fato de basear-se na aleatoriedade inicial das soluções, o que também implica que execuções distintas sobre a mesma base de dados retorne uma soluções diferentes.

\section*{Resultado da execução - O Algoritmo e sua Ação sobre as Instâncias} \label{sec4}
A aplicação do JPSO para PAGRM (que é um problema NP completo \cite{}) sobre a base disponibilizada gerou dados de saída que foram organizados em um arquivo texto em formato \textit{txt}. Eles podem ser verificados de forma organizada nas tabelas a seguir em \ref{tab1}, \ref{tab2}, \ref{tab3} e \ref{tab4} e posteriormente são comparados aos resultados dos algoritmos exibidos em \cite{}, estes sobre a mesma base de dados. 
\begin{table}[!h]
	

\begin{tabular}{ccccc}

	\hline \rule[-2ex]{0pt}{5.5ex} N & L & D & F. Objetiva & Tempo (ms) \\ 
	\hline \rule[-2ex]{0pt}{5.5ex} 20 & 20 & 0.8 & 2.4 & 1002 \\ 
	 \rule[-2ex]{0pt}{5.5ex}  &  & 0.5 & 3.1 & 1543 \\ 
	 \rule[-2ex]{0pt}{5.5ex}  &  & 0.2 & 6.8 & 2156 \\ 
	 \rule[-2ex]{0pt}{5.5ex} 30 & 30 & 0.8 & 2.8 & 1890 \\ 
	 \rule[-2ex]{0pt}{5.5ex}  &  & 0.5 & 3.7 & 4291 \\ 
	 \rule[-2ex]{0pt}{5.5ex}  &  & 0.2 & 7.5 & 6191 \\ 
	 \rule[-2ex]{0pt}{5.5ex} 40 & 40 & 0.8 & 2.9 & 3021 \\ 
	 \rule[-2ex]{0pt}{5.5ex}  &  & 0.5 & 3.7 & 8018 \\ 
	 \rule[-2ex]{0pt}{5.5ex}  &  & 0.2 & 7.7 & 14583 \\ 
	 \rule[-2ex]{0pt}{5.5ex} 50 & 50 & 0.8 & 3.0 & 6729 \\ 
	 \rule[-2ex]{0pt}{5.5ex}  &  & 0.5 & 4.5 & 15001 \\ 
	 \rule[-2ex]{0pt}{5.5ex}  &  & 0.2 & 8.8 & 3113 \\ 
	\hline \rule[-2ex]{0pt}{5.5ex}  &  & Total & 56.9 & 67538 \\ 
	\hline 
\end{tabular} 

\caption{N é o número de nós, L é o número de rótulos e D é a densidade.  A tabela mostra o tempo de execução de cada}
\label{tab1}
\end{table}
\begin{table}[!h]


\begin{tabular}{ccccc}
	\hline \rule[-2ex]{0pt}{5.5ex} N & L & D & F. Objetiva & Tempo (ms) \\ 
	 \hline\rule[-2ex]{0pt}{5.5ex} 100 & 25 & 0.8 & 1.8 & 11440 \\ 
	 \rule[-2ex]{0pt}{5.5ex}  &  & 0.5 & 2.0 & 16664 \\ 
	 \rule[-2ex]{0pt}{5.5ex}  &  & 0.2 & 4.6 & 46836 \\ 
	 \rule[-2ex]{0pt}{5.5ex}  & 50 & 0.8 & 2.0 & 31146 \\ 
	 \rule[-2ex]{0pt}{5.5ex}  &  & 0.5 & 3.0 & 34667 \\ 
	 \rule[-2ex]{0pt}{5.5ex}  &  & 0.2 & 6.9 & 107502 \\ 
	 \rule[-2ex]{0pt}{5.5ex}  & 100 & 0.8 & 3.8 & 37138 \\ 
	 \rule[-2ex]{0pt}{5.5ex}  &  & 0.5 & 5.3 & 96218 \\ 
	 \rule[-2ex]{0pt}{5.5ex}  &  & 0.2 & 11.3 & 192166 \\ 
	 \rule[-2ex]{0pt}{5.5ex}  & 125 & 0.8 & 4.1 & 60090 \\ 
	 \rule[-2ex]{0pt}{5.5ex}  &  & 0.5 & 6.2 & 113139 \\ 
	 \rule[-2ex]{0pt}{5.5ex}  &  & 0.2 & 13.1 & 239019 \\ 
	 \hline\rule[-2ex]{0pt}{5.5ex}  & Total -- &  & 64.1 & 986025 \\ 
	\hline 
\end{tabular} 
\caption{N é o número de nós, L é o número de rótulos e D é a densidade.  A tabela mostra o tempo de execução de cada}
\label{tab2}
\end{table}



\begin{table}[!h]


\begin{tabular}{ccccc}
	\hline \rule[-2ex]{0pt}{5.5ex} N & L & D  & F. Objetiva & Tempo (ms) \\ 
	\hline \rule[-2ex]{0pt}{5.5ex} 200 & 50 & 0.8 & 2.0 & 53185 \\ 
	\rule[-2ex]{0pt}{5.5ex}  &  & 0.5 & 2.3 & 130555 \\ 
	 \rule[-2ex]{0pt}{5.5ex}  &  & 0.2 & 5.7 & 306403 \\ 
	 \rule[-2ex]{0pt}{5.5ex}  & 100 & 0.8 & 2.9 & 99256 \\ 
	 \rule[-2ex]{0pt}{5.5ex}  &  & 0.5 & 4.1 & 157123 \\ 
	 \rule[-2ex]{0pt}{5.5ex}  &  & 0.2 & 9.4 & 574387 \\ 
	 \rule[-2ex]{0pt}{5.5ex}  & 200 & 0.8 & 4.6 & 254051 \\ 
	 \rule[-2ex]{0pt}{5.5ex}  &  & 0.5 & 6.8 & 655515 \\ 
	 \rule[-2ex]{0pt}{5.5ex}  &  & 0.2 & 15.3 & 1184675 \\ 
	 \rule[-2ex]{0pt}{5.5ex}  & 250 & 0.8 & 5.1 & 566244 \\ 
	 \rule[-2ex]{0pt}{5.5ex}  &  & 0.5 & 8.0 & 933404 \\ 
	 \rule[-2ex]{0pt}{5.5ex}  &  & 0.2 & 17.9 & 1441719 \\ 
	\hline \rule[-2ex]{0pt}{5.5ex}  &  & Total --  & 84.1 & 6856514 \\ 
	\hline 
\end{tabular} 
\caption{N é o número de nós, L é o número de rótulos e D é a densidade.  A tabela mostra o tempo de execução de cada}
\label{tab3}
\end{table}

\begin{table}[!h]
	\begin{tabular}{ccccc}
		\hline \rule[-2ex]{0pt}{5.5ex} N & L & D  & F. Objetiva & Tempo (ms) \\ 
		\hline \rule[-2ex]{0pt}{5.5ex} 500 & 125 & 0.8 & 2.0 & 580854 \\ 
		\hline \rule[-2ex]{0pt}{5.5ex}  &  & 0.5 & 3.1 & 1535064 \\ 
		\hline \rule[-2ex]{0pt}{5.5ex}  &  & 0.2 & 7.3 & 4508425 \\ 
		\hline \rule[-2ex]{0pt}{5.5ex}  & 250 & 0.8 & 3.1 & 2351650 \\ 
		\hline \rule[-2ex]{0pt}{5.5ex}  &  & 0.5 & 4.9 & 6289072 \\ 
		\hline \rule[-2ex]{0pt}{5.5ex}  &  & 0.2 & 12.9 & 9052493 \\ 
		\hline \rule[-2ex]{0pt}{5.5ex}  & 500 & 0.8 & 5.8 & 6751334 \\ 
		\hline \rule[-2ex]{0pt}{5.5ex}  &  & 0.5 & 8.7 & 16560601 \\ 
		\hline \rule[-2ex]{0pt}{5.5ex}  &  & 0.2 & 21.3 & 16661087 \\ 
		\hline \rule[-2ex]{0pt}{5.5ex}  & 625 & 0.8 & 6.7 & 18957465 \\ 
		\hline \rule[-2ex]{0pt}{5.5ex}  &  & 0.5 & 10.5 & 17878234 \\ 
		\hline \rule[-2ex]{0pt}{5.5ex}  &  & 0.2 & 25.4 & 18310678 \\ 
		\hline \rule[-2ex]{0pt}{5.5ex}  &  & Total -- &  &  \\ 
		\hline 
	\end{tabular}  
	\caption{N é o número de nós, L é o número de rótulos e D é a densidade.  A tabela mostra o tempo de execução de cada}
	\label{tab4}
\end{table}
\section{Comparação de Resultados} \label{sec5}
Dentre as tarefas necessárias para a realização deste trabalho está a realização da comparação entre os algoritmos presentes em \cite{} e o escolhido para a realização de implementação. Em \cite{} os algoritmos discutidos e executados sobre a mesma base de dados utilizada neste trabalho são: EXACT, PILOT, MGA, GRASP e VNS. Assim, como os algoritmos são aplicados em uma base comum a deste trabalho, é possível estabelecer uma comparação plausível. 

Tal comparação está divida em 3 grupos, baseados nas características de suas instâncias: grupo 1 é referente a instâncias pequenas que possuem número de vértices iguais ao número de rótulos, o grupo 2 considera instâncias maiores com o número de rótulos definidos por $0,25 * n$, onde $n$ é o número de vértices e os rótulos varia de 25 a 125. Por último, o grupo 3 são as maiores instâncias presentes na base, pois é composta por parâmetros como número de vértices igual a 200 e número de rótulos que variam de 50 a 250 como visto na tabela \ref{tab3} e número de vértices igual a 500 com número de rótulos variando de 125 a 625 como visto na tabela \ref{tab4}.
\subsection{Comparações no grupo 1}
No grupo 1 (120 instâncias), dentre os algoritmos apresentados em \cite{} os algoritmos MGA e PILOT são antagônicos: O segundo é mais rápido que o primeiro, entretanto provendo respostas piores. Enquanto isso, o segundo é mais demorado quanto a execução, mas gera soluções melhores. Por sua vez, GRASP e VNS são os mais rápidos e promovem a solução exata. Estes são excelente para essas instâncias. Comparando o JPSO implementado é perceptível que este é mais rápido que MGA (em função do tempo de execução) e mais lento que os demais. Além disso, o JPSO não promove a certeza de que a solução obtida é a exata (ótima). 

O MGA é o mais lento por ser o que mais diversifica as possíveis soluções (no objetivo de encontrar a melhor), porém retornando uma resposta melhor que o PILOT que
diversifica pouco, mas apresenta uma resposta pior. Assim, o JPSO é mais executa mais rápido do que o algoritmo mais lento apresentado em \cite{} , levando em considerações as instâncias do grupo 1 (cujas características podem ser vistas na tabela \ref{tab1}).



\subsection{Comparações no grupo 2}
	No grupo 2, instâncias com número de vértices igual a 100 (360 instâncias), a melhor performance é tida pelo algoritmo VNS. Isto, pois apresenta como solução a melhor resposta (exata) e no menor tempo dentre os algoritmos apresentados. GRASP por sua vez também pode ser enquadrado como um algoritmo bom e obtém a mesma solução que o VNS. Aqui PILOT apresenta o segundo maior tempo de execução, tendo respostas piores que os demais. MGA continua sendo aquele de maior tempo. Aqui é possível perceber que ao aumentar o número de instâncias, o PILOT passa a perder aquilo que ele tinha de vantagem em relação aos demais que é a rapidez de execução. O JPSO aqui implementado possui um tempo de maior e ainda não produz uma resposta com a garantia que seja a solução ótima. Portanto a melhor opção para resolver um PAGRM é o uso do VNS.

\subsection{Comparações no grupo 3}
Para instâncias maiores, como as que compõem o grupo 3 (referente à tabela \ref{tab3} e \ref{tab4}) as heurísticas referentes ao VNS e ao GRASP se mantiveram superiores nos quesitos qualidade da resposta e tempo de execução. O GRASP passa a ter menor tempo de execução em relação ao VNS quão maior for o PGRM em questão. Em relação ao JPSO, é perceptível através da tabela \ref{tab3} que em instâncias com o número de vértices igual a 200 possui um tempo de execução melhor apenas que o MGA. já em instâncias com o número de vértices igual a 500 (as maiores instâncias) é notável que ele apresenta o maior tempo de execução sem garantir o resultado ótimo.

 
\section{Conclusão} \label{sec6}
Em nosso trabalho implementamos o algoritmo JPSO para fim de execução da base de dados disponibilizada. Esta é a mesma aplicada em \cite{} e este fato permitiu que fossem comparados os resultados obtidos pela implementação aqui discutida e aqueles apresentados por demais soluções. O objetivos deste trabalho foi ampliar os conhecimentos a respeito do PAGRM que é o foco dos algoritmos apresentados. 

Nota-se que para instâncias maiores os algoritmos VNS e GRASP são os mais compensativos, levando em consideração aspectos como tempo de execução e garantia de solução ótima. O JPSO implementado possui um baixo rendimento para resolver um JPSO quando comparado às duas melhores heurísticas, pois não garante a obtenção da solução exata e ainda tem um tempo de execução maior, sendo compensativo em alguns casos em relação ao MGA e ao PILOT.  Acredita-se que o motivo do execução do JPSO implementado está atrelado ao critério de parada do laço de repetição contido no algoritmo: A execução continuará até que a melhor solução global não se altere por até 100 iterações, assim a redução desse critério influencia no tempo de execução. Além disso, foi possível notar que dependendo do objetivo e das condições apresentadas em mundo real, uma solução rápida de ser executada pode gerar uma resposta que não compensa -- caso do PILOT em um conjunto de instâncias pequenas. Este trabalho foi fundamental para ampliar a visão a respeito do PAGRM e a forma como especialistas da área de grafos estudam heurísticas candidatas a serem boas alternativas de resolução.
%\end{document}  % This is where a 'short' article might terminate

.

%
% The following two commands are all you need in thee
% initial runs of your .tex file to
% produce the bibliography for the citations in your paper.
\bibliographystyle{abbrv}
\bibliography{sigproc}  % sigproc.bib is the name of the Bibliography in this case
% You must have a proper ".bib" file
%  and remember to run:
% latex bibtex latex latex
% to resolve all references
%
% ACM needs 'a single self-contained file'!
%
%APPENDICES are optional
%\balancecolumns
%\appendix
%Appendix A


\end{document}
